\chapter*{Conclusione} %Se si cambia il Titolo cambiare anche la riga successiva così che appia corretto nell'conclusione
\addcontentsline{toc}{chapter}{Conclusione} %Per far apparire Introduzione nell'indice (Il nome deve rispecchiare quello del chapter)
Questo esercizio consiste in un semplice programma con i Thread, in particolare è pensato come primo impatto
alla progammazione multi Thread in cui si cerca di fornire un primo approccio con esso, attraverso i metodi 
classici: Notift(), NotifyAll() e Wait().
Si cerca, inoltre, di insegnare al lettore la metodologia corretta, in quanto lo stesso programma lo si poteva realizzare anche in altri modi, per esempio attraverso l'utilizzo di più classi, inserendo il metodo ``doWait()" e ``doNotify()" in 2
classi separate (approccio dal punto di vista metodologico scorretto).
Infine, si cerca anche di fornire familiarità con l'utilizzo del blocco ``syncronized \{ ... \}", in cui vengono
inserite in esso quelle risorse in mutua esclusione che non devono essere accedute/modificate da 2 o più Thread 
contemporaneamente.